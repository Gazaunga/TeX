\documentclass[11pt,open=any]{scrbook}              % Book class in 11 points
\usepackage[Bjornstrup]{fncychap}
\usepackage{xcolor}
% Removes older commands that prevent koma classes from working with Overleaf
\makeatletter
\DeclareOldFontCommand{\rm}{\normalfont\rmfamily}{\mathrm}
\DeclareOldFontCommand{\sf}{\normalfont\sffamily}{\mathsf}
\DeclareOldFontCommand{\tt}{\normalfont\ttfamily}{\mathtt}
\DeclareOldFontCommand{\bf}{\normalfont\bfseries}{\mathbf}
\DeclareOldFontCommand{\it}{\normalfont\itshape}{\mathit}
\DeclareOldFontCommand{\sl}{\normalfont\slshape}{\@nomath\sl}
\DeclareOldFontCommand{\sc}{\normalfont\scshape}{\@nomath\sc}
\makeatother
% 

\colorlet{partbgcolor}{gray!30}% shaded background color for parts
\colorlet{partnumcolor}{gray}% color for numbers in parts
\colorlet{chapbgcolor}{gray!30}% shaded background color for chapters
\colorlet{chapnumcolor}{gray}% color for numbers in chapters

\renewcommand*\partformat{%
  \fontsize{76}{80}\usefont{T1}{pzc}{m}{n}\selectfont%
  \hfill\textcolor{partnumcolor}{\thepart}}

\makeatletter
\renewcommand*{\@part}{}
\def\@part[#1]#2{%
  \ifnum \c@secnumdepth >-2\relax
    \refstepcounter{part}%
    \@maybeautodot\thepart%
    \addparttocentry{\thepart}{#1}%
  \else
    \addparttocentry{}{#1}%
  \fi
  \begingroup
    \setparsizes{\z@}{\z@}{\z@\@plus 1fil}\par@updaterelative
    \raggedpart
    \interlinepenalty \@M
    \normalfont\sectfont\nobreak
    \setlength\fboxsep{0pt}
    \colorbox{partbgcolor}{\rule{0pt}{40pt}%
    \makebox[\linewidth]{%
    \begin{minipage}{\dimexpr\linewidth+20pt\relax}
      \ifnum \c@secnumdepth >-2\relax
        \vskip-25pt
        \size@partnumber{\partformat}%
      \fi      %
      \vskip\baselineskip
      \hspace*{\dimexpr\myhi+10pt\relax}%
      \parbox{\dimexpr\linewidth-2\myhi-20pt\relax}{\raggedleft\LARGE#2\strut}%
      \hspace*{\myhi}\par\medskip%
    \end{minipage}%
      }%
    }%
    \partmark{#1}\par
  \endgroup
  \@endpart
}

\renewcommand\DOCH{%
  \settowidth{\py}{\CNoV\thechapter}
  \addtolength{\py}{-10pt}
  \fboxsep=0pt%
  \colorbox{chapbgcolor}{\rule{0pt}{40pt}\parbox[b]{\textwidth}{\hfill}}%
  \kern-\py\raise20pt%
  \hbox{\color{chapnumcolor}\CNoV\thechapter}\\%
}

\renewcommand\DOTI[1]{%
  \nointerlineskip\raggedright%
  \fboxsep=\myhi%
  \vskip-1ex%
  \colorbox{chapbgcolor}{\parbox[t]{\mylen}{\CTV\FmTi{#1}}}\par\nobreak%
  \vskip 40pt%
}

\renewcommand\DOTIS[1]{%
  \fboxsep=0pt
  \colorbox{chapbgcolor}{\rule{0pt}{40pt}\parbox[b]{\textwidth}{\hfill}}\\%
  \nointerlineskip\raggedright%
  \fboxsep=\myhi%
  \colorbox{chapbgcolor}{\parbox[t]{\mylen}{\CTV\FmTi{#1}}}\par\nobreak%
  \vskip 40pt%
 }
\makeatother
%%

\parindent0pt  \parskip10pt             % make block paragraphs
\raggedright                            % do not right justify

\title{\bf An Example of Book Class}    % Supply information
\author{for \LaTeX\ Class}              %   for the title page.
\date{\today}                           %   Use current date. 

% Note that book class by default is formatted to be printed back-to-back.
%-------------------
\author{J.M. Ottley}
\title{Book!}
%ADDITIONAL INFO FOR PDF
%INFO DLA PDF
\pdfinfo {
/Author (\author)
/Subject (ebook,)
/Title (\title)
/Creator (TexLive&Linux)
/Producer (pdflatex)
/CreationDate (D:20121212121212)
%/ModDate (D:\pdfdate)
/Keywords (ebook;6inches)
}

\begin{document}                        % End of preamble, start of text.
\frontmatter                            % only in book class (roman page #s)
\maketitle                              % Print title page.
\tableofcontents                        % Print table of contents
\mainmatter                             % only in book class (arabic page #s)
\part{A Part Heading}                   % Print a "part" heading
\chapter{A Main Heading}                % Print a "chapter" heading
Most of this example applies to \texttt{article} and \texttt{book} classes
as well as to \texttt{report} class. In \texttt{article} class, however,
the default position for the title information is at the top of
the first text page rather than on a separate page. Also, it is
not usual to request a table of contents with \texttt{article} class.
 
\section{A Subheading}                  % Print a "section" heading
The following sectioning commands are available:
\begin{quote}                           % The following text will be
 part \\                                %    set off and indented.
 chapter \\                             % \\ forces a new line
 section \\ 
 subsection \\ 
 subsubsection \\ 
 paragraph \\ 
 subparagraph 
\end{quote}                             % End of indented text
But note that---unlike the \texttt{book} and \texttt{report} classes---the
\texttt{article} class does not have a ``chapter" command.
 
\begin{thebibliography}{99}
  \addcontentsline{toc}{chapter}{Bibliography}
\bibitem{lamport} L. Lamport. {\bf \LaTeX \ A Document Preparation System}
Addison-Wesley, California 1986.
\end{thebibliography}

\end{document}                          % The required last line
